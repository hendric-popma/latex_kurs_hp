\chapter{Beispiele}
In folgendem Abschnitt sind Beispiele für den \enquote{Code} dargestellt.\\

\section{Kapitel erstellen und Label einfügen}
	\label{sec:kapitel_erstellen} % hier erstelle ich ein label für das Kapitel und kann damit darauf referenzieren

Bsp.: In Kapitel \ref{sec:kapitel_erstellen} sehen wir uns an wie man  Kapitel erstellt.
% Befehle um Kapitel und Sections zu erstellen
%\chapter{Kapitelüberschrift}
%\section{Abschnitssüberschrift}
%\subsection{UnterAbschnitt}
%\subsubsection{tUnterUnterAbschnitt}

%TODO Grafik einbinden
\subsection{Einbinden von Grafiken}
\begin{figure}[H]
	\centering
%	\includegraphics[width=0.9\linewidth]{einbindung_beispielgrafik.pdf}
	\caption[Titel fürs Inhaltsverzeichnis]{Titel unter Grafik mit Quelle}
	\cite{agfw_leitfaden_wp}
	\label{fig:grafik1}
\end{figure}


\subsubsection{Bild neben Text}
\begin{figure}[H]
	\begin{minipage}[t]{0.49\textwidth}
		Das hier ist der Beispieltext \\
		er kann über mehrere Zeilen gehen\\
		man kann die Höhe einstellen über \enquote{textwidth}
	\end{minipage}
	\hfill%
	\begin{minipage}[t]{0.49\textwidth}% adapt widths of minipages to your needs
		%		abbildung
		\vspace{-3mm}
%		\includegraphics[width=\textwidth]{einbindung_beispielgrafik.pdf}
		\caption{Wärmeverteilung für Szenario 1} % caption für beides Zusammen
		\label{fig:grafik2}
	\end{minipage}
\end{figure}


\section{Aufzählungen erstellen}
Es können verschiedene Aufzählungsarten verwendet werden.
\begin{itemize}[noitemsep]
	\item Wärmepumpe
	\item Gaskessel 
	\item BHKW
	\item Gasturbine
\end{itemize}

\begin{itemize}[noitemsep]
	\item [] Wärmepumpe
	\item [] Gaskessel 
	\item [] BHKW
	\item [] Gasturbine
\end{itemize}

%\begin{itemize}
%	\item [] Wärmepumpe
%	\item [] Gaskessel 
%	\item [] BHKW
%	\item [] Gasturbine
%\end{itemize}

%\begin{itemize}[noitemsep, label=-]
%	\item Wärmepumpe
%	\item Gaskessel 
%	\item BHKW
%	\item Gasturbine
%\end{itemize}

%label options z.B.  \Roman*, \Alph*
\begin{enumerate}[noitemsep]
	\item Wärmepumpe
	\item Gaskessel 
	\item BHKW
	\item Gasturbine
\end{enumerate}

\begin{enumerate}[noitemsep, label=\alph*]
	\item Wärmepumpe
	\item Gaskessel 
	\item BHKW
	\item Gasturbine
\end{enumerate}




\section{Sonstige Befehle}
\href{https://forschung.hs-kempten.de/de/forschungsprojekt/482-heatshift}{\enquote{ HeatSHIFT}}
CO\textsubscript{2}-Kosten

\section{Tabellen erstellen}
Hier erstellen wir eine einfache Tabelle mit dem Paket \textbf{tabular}\\
\begin{table}[H]
	\centering
	\caption[Überschrift für Tabellenverzeichnis]{Überschrift der Tabelle}
	\label{tab:tabelle_einfach}
	\begin{tabular}{|l|c|r|}\hline
		Anlage&Wert&Einheit\\\hline
		Wärmepumpe&10&MW\\\hline
		Gaskessel&40&MW\\\hline
	\end{tabular}
\end{table}

\begin{table}[H]
	\centering
	\caption[Überschrift für Tabellenverzeichnis]{Überschrift der Tabelle}
	\label{tab:tabelle_einfach}
	\begin{tabular}{|l|c|r|}\hline
		\multicolumn{1}{|c|}{\textbf{Anlage}}&\textbf{max. Leistung}&\textbf{Einheit}\\\hline
		MHKW&20 & MW\\\hline
		HHKW&12 & MW\\\hline
		Spitzenlast 1 (Gas- bzw. Ölkessel)& 2 x 7,5 & MW\\\hline
		Spitzenlast 2 (Ölkessel)& 3 x 10 & MW\\\hline
		Wärmespeicher be-/entladen& 10 & MW\\\hline
	\end{tabular}
\end{table}

Hier wird es komplizierter. Es wird eine Tabelle mit dem Paket \textbf{tabularx} erstellt\\
\begin{table}[H]
	\newcolumntype{Z}{>{\centering \arraybackslash}X}
	\renewcommand{\arraystretch}{1}
	\scriptsize
	\caption{Auswertung aller Szenarien mit Variante 1 und 5}\label{tab:auswertung_alle_1}
	\rotatebox{90}{
		\begin{tabularx}{\textheight}{|l|c|Z|Z|Z|Z|Z|Z|Z|Z|Z|}\hline
			&\textbf{Ausgangsszenario}&\multicolumn{3}{c|}{\textbf{Szenario 1.1\footnotemark}}&\multicolumn{3}{c|}{\textbf{Szenario 1.5\footnotemark}}&\multicolumn{3}{c|}{\textbf{Szenario 2.1\footnotemark}}\\\hline
			&Ergebnis \mbox{in \euro}&Ergebnis \mbox{in \euro}&Einsparung in \euro&Einsparung in \%&Ergebnis \mbox{in \euro}& Einsparung in \euro&Einsparung in \%&Ergebnis \mbox{in \euro}&Einsparung in \euro&Einsparung in \%\\
			\hline
			\hline
			\textbf{Gesamtkosten} &  14.615.186  &  22.605.542  &  -7.990.357  & -54,67 &  21.949.419  &  -7.334.234  & -50,18 &  22.454.659  &  -7.839.474  & -53,64 \\
			\hline
			\rowcolor{lightgray}- Investitionskosten &  12.150.000  &  20.400.000  &  -8.250.000  & -67,90 &  20.400.000  &  -8.250.000  & -67,90 &  20.400.000  &  -8.250.000  & -67,90 \\
			\hline
			\rowcolor{lightgray}- variable Betriebskosten &  2.345.556  &  2.022.111  & 323.445  & 13,79 &  1.365.988  & 979.569  & 41,76 &  1.871.228  & 474.329  & 20,22 \\
			\hline
			\quad- CO\textsubscript{2}-Kosten& 173.343  & 97.490 &  75.852 & 43,76 &  18.442&  154.901 & 89,36 &78.533 & 94.810  & 54,70\\
			\hline
			\quad - Netznutzungskosten  &  303.508  &  458.897  &  -155.389  & -51,20 &  385.385  &  -81.877  & -26,98 &  497.721  &  -194.212  & -63,99 \\
			%	\hline
			\quad \quad - Gas &  &  170.697  &  &  &  32.290  &  &  &  137.504  &  &  \\
			%	\hline
			\quad \quad - Strom &  &  288.200  &  &  &  353.095  &  &  &  360.217  &  &  \\
			\hline
			\quad - Bereitstellungskosten &  1.797.908  &  1.415.654  & 382.254  & 21,26 &  942.068  & 855.840  & 47,60 &  1.250.086  & 547.822  & 30,47 \\
			%	\hline
			\quad \quad - Gas &  &  630.612  &  &  &  190.778  &  &  &  484.060  &  &  \\
			%	\hline
			\quad \quad - Strom &  &  785.042  &  &  &  751.290  &  &  &  766.026  &  &  \\
			\hline
			\quad - sonstige var. Kosten &  70.797  &  50.069  & 20.728  & 29,28 &  20.093  & 50.705  & 71,62 &  44.889  & 25.909  & 36,60 \\
			%	\hline
			\quad \quad - Gas &  &  39.817  &  &  &  7.532  &  &  &  32.075  &  &  \\
			%	\hline
			\quad \quad - Strom &  &  10.252  &  &  &  12.560  &  &  &  12.814  &  &  \\
			\hline
			\rowcolor{lightgray}- fixe Betriebskosten  &  119.629  &  183.432  &  -63.802  & -53,33 &  183.432  &  -63.802  & -53,33 &  183.432  &  -63.802  & -53,33 \\
			\hline
			\multicolumn{9}{l}{}\\
			\hline
			&Ergebnis&Ergebnis&Einsparung gesamt&Einsparung in \%&Ergebnis&Einsparung gesamt&Einsparung in \%&Ergebnis&Einsparung gesamt&Einsparung in \%\\
			\hline
			variable Wärmegestehungskosten \footnotemark &  78 \euro/MWh  &  67 \euro/MWh  &  11 \euro/MWh  & 13,79
			&  45 \euro/MWh  &  32 \euro/MWh  & 41,76
			&  62 \euro/MWh  &  16 \euro/MWh & 20,22
			\\
			\hline
			Betriebsstunden Gaskessel &  2.133 h  &  1.033 h  &  1.100 h  & 51,57
			&  266 h  &  1.867 h  & 87,53
			&  777 h  &  1.356 h  & 63,57
			\\
			\hline
			CO\textsubscript{2}-Ausstoß &  5.778\,t\textsubscript{CO2}  &  4.860\,t\textsubscript{CO2}  & 918\,t\textsubscript{CO2}  & 15,89 &  2.588\,t\textsubscript{CO2}  & 3.190\,t\textsubscript{CO2}  & 55,21 &  4.631\,t\textsubscript{CO2}&  1.148\,t\textsubscript{CO2}& 19,86\\
			\hline
			\multicolumn{9}{l}{\footnotemark[1]{Szenario 1.1: Gaskessel: 45 MW; WP: 20 MW, COP: 3; Speicher: 40 MWh, 10 MW}}\\
			\multicolumn{9}{l}{\footnotemark[2]{Szenario 1.5: Gaskessel: 45 MW; WP: 20 MW, COP: 5; Speicher: 40 MWh, 10 MW}}\\
			\multicolumn{9}{l}{\footnotemark[3]{Szenario 2.1: Gaskessel: 45 MW; WP: 20 MW, COP: 3; Speicher: 80 MWh, 20 MW}}\\
			\multicolumn{9}{l}{\footnotemark[4]{Berechnung: Erzeugte Wärmemenge geteilt durch variable Betriebskosten}}\\
			
		\end{tabularx}
	}
\end{table}



\begin{table}[H]
	\newcolumntype{Z}{>{\centering \arraybackslash}X}
	\caption{Wärmegestehungskosten (LCoH) über verschiedene Zeiträume}
	\label{tab:lcoh}
	\renewcommand{\arraystretch}{1}
	\scriptsize
	%	\rotatebox{90}{
		\begin{tabularx}{\textwidth}{|Z|Z|c|c|c|c|c|}
			\hline
			&&\textbf{Ausgangsszenario}&\multicolumn{2}{c|}{\textbf{Szenario 1.1}}& \multicolumn{2}{c|}{\textbf{Szenario 3.5}}\\\hline
			&Zinssatz & Invest.\,100\,\% &Invest.\,100\,\%&  Invest.\,60\,\% & Invest.\,100\,\%&  Invest.\,60\,\%\\
			\hline
			\hline
			\multirow{3}{*}{\shortstack[c]{ \textbf{LCoH }\\ \textbf{15 Jahre}\\in \euro/MWh}} & 3\,\% & 115,48		&	129,79	&	107,12	&	101,41	&	78,75
			\\
			\cline{2-7}
			& 5\,\% & 120,55		&	138,29	&	112,23	&	109,92	&	83,85
			\\
			\cline{2-7}
			& 8\,\% & 128,80		&	152,15	&	120,54	&	123,77	&	92,16
			\\
			\hline
			\hline
			\multirow{3}{*}{\shortstack[c]{\textbf{LCoH }\\ \textbf{20 Jahre}\\in \euro/MWh}} & 3\,\% & 108,81		&	118,59	&	100,41	&	90,21	&	72,03
			\\
			\cline{2-7}
			& 5\,\% & 114,06		&	127,40	&	105,69	&	99,03	&	77,32
			\\
			\cline{2-7}
			& 8\,\% & 122,77		&	142,02	&	114,46	&	113,64	&	86,09
			\\
			\hline
			\hline
			\multirow{3}{*}{\shortstack[c]{ \textbf{LCoH }\\ \textbf{25 Jahre}\\in \euro/MWh}}  & 3\,\% & 104,87		&	111,97	&	96,43	&	83,59	&	68,06
			\\
			\cline{2-7}
			& 5\,\% & 110,32		&	121,12	&	101,92	&	92,74	&	73,55
			\\
			\cline{2-7}
			& 8\,\% & 119,47		&	136,49	&	111,14	&	108,11	&	82,77 \\
			\hline
		\end{tabularx}
		%}
\end{table}


\section{Gleichungen erstellen}
\begin{equation}
	\label{eq:gesamtkosten}
	\textrm{Gesamtkosten} = K^{Invest} + K^{fix. Betrieb} + K^{var. Betrieb}
\end{equation}
\begin{equation}
	\label{eq:investitionskosten}
	K^{Invest} = \dot{Q}^{Gas, max} \cdot k^{Invest, Gas} + \dot{Q}^{WP, max} \cdot k^{Invest, WP}
\end{equation}
\begin{equation}
	\label{eq:fix_betriebskosten}
	K^{fix.Betrieb} = \dot{Q}^{Gas, max} \cdot k^{fix. Betrieb, Gas} + \dot{Q}^{WP, max} \cdot k^{fix. Betrieb, WP}
\end{equation}


\begin{align}
	&\dot{Q}^{Last}_{t} = \dot{Q}^{Gaskessel}_{t} \cdot on^{Gas}_{t} + \dot{Q}^{WP}_{t} \cdot on^{WP}_{t} + \dot{Q}^{entladen}_{t} - \dot{Q}^{laden}_{t}&\forall t \in T \label{eq:opt_energie_netz}
\end{align}
Die Nebenbedingungen in Gleichung~\ref{eq:opt_min_max_wp} und \ref{eq:opt_min_max_gas} regulieren die minimale und maximale Erzeugerleistung der Wärmepumpe und des Gaskessels.
\begin{align}
	&\dot{Q}^{WP,min} \leq \dot{Q}^{WP}_{t} \leq \dot{Q}^{WP,max}&\forall t \in T \label{eq:opt_min_max_wp}\\
	&\dot{Q}^{Gas,min} \leq \dot{Q}^{Gas}_{t} \leq \dot{Q}^{Gas,max}&\forall t \in T \label{eq:opt_min_max_gas}
\end{align}
Das Verhalten des Kurzzeitwärmespeichers wird durch folgende Nebenbedingungen abgebildet. Dabei wird in Gleichung~\ref{eq:opt_grenze_beladen_speicher} und \ref{eq:opt_grenze_entladen_speicher} die Be- und Entladeleistung begrenzt, in Gleichung~\ref{eq:opt_start_speicher} und \ref{eq:opt_end_speicher} wird der Inhalt des Speichers zum Start der Optimierung festgelegt sowie dass der Speicher am Ende mindestens den gleichen Inhalt wie am Start aufweist.
\begin{align}
	&\dot{Q}^{Speicher,min} \leq \dot{Q}^{laden}_{t} \leq \dot{Q}^{Speicher,max}&\forall t \in T \label{eq:opt_grenze_beladen_speicher}\\
	&\dot{Q}^{Speicher,min} \leq \dot{Q}^{entladen}_{t} \leq \dot{Q}^{Speicher,max} &\forall t \in T \label{eq:opt_grenze_entladen_speicher}\\
	&Q^{Speicher}_{t=0} = Q^{init, sp} \cdot Q^{Kapzit \ddot{a} t}& \label{eq:opt_start_speicher}\\
	&Q^{Speicher}_{t=8760} \geq Q^{init, sp} \cdot Q^{Kapzit \ddot{a} t} & \label{eq:opt_end_speicher}
\end{align}
Des Weiteren wird in der Nebenbedingung mit der Gleichung~\ref{eq:opt_energie_speicher} die gespeicherte Energie berechnet. Dazu wird die Energie des vorherigen Zeitschrittes ($t-1$) mit der zugeführten und abgeführten Energie im Zeitschritt ($t$) verrechnet.
\begin{align}	
	&Q^{Speicher}_{t}=Q^{Speicher}_{t-1} + \dot{Q}^{beladen}_{t} \cdot \Delta t - \dot{Q}^{entladen}_{t} \cdot \Delta t &\forall t \in T \label{eq:opt_energie_speicher}
\end{align}
Außerdem werden die physikalischen Eigenschaften für das Be- und Entladen des Speichers in den Gleichungen~\ref{eq:opt_laden_speicher} bis \ref{eq:opt_entladen_speicher} abgebildet. Diese Nebenbedingungen sorgen dafür, dass nicht mehr Energie entnommen wird als im Speicher enthalten ist und dass die Beladeleistung nicht größer ist als die von den Erzeugern zugeführte Energie.
\begin{align}
	&\dot{Q}^{laden}_{t} \leq \dot{Q}^{Gas}_{t} \cdot On^{Gas}_{t} + \dot{Q}^{WP}_{t} \cdot On^{WP}_{t} \label{eq:opt_laden_speicher}&\forall t \in T\\
	&\dot{Q}^{entladen}_{t} \cdot \Delta t \leq Q^{Speicher}_t &\forall t \in T \label{eq:opt_entladen_speicher}
\end{align}


\section{Wie zitiere ich in LaTex}
Zitate können ganz einfach am Satzende über den Befehl \textbf{\textbackslash cite\{\}} eingebunden werden.\\
Bsp.: Zur erfolgreichen Einbindung einer Großwärmepumpe in ein Wärmenetz muss eine geeignete Abwärmequelle, wie z.B. Abwärme aus einem Industrieprozess oder Flusswasser zur Verfügung stehen \cite{agfw_leitfaden_wp}.\\
Ich kann auch zwei Quellen am Satzende einfügen.\\
Der Handel am Spot-Markt ist eingeteilt in den Day-Ahead und Intraday Handel \cite{next_strommarkt, duis_future_options}


\section{Übungen}

\begin{table}[H]
	\centering
	\caption{Dies ist die zweite Tabelle}
	\label{tab:uebung1}
	\begin{tabular}{l|c|r}
		Bezeichnung&Werte&Einheit \\\hline
		Gaskraftwerk&10&MW\\
		Wärmepumpe&5&MW\\
	\end{tabular}
\end{table}

\begin{table}[H]
	\newcolumntype{Z}{>{\centering \arraybackslash}X}
	\caption{Dies ist die zweite Tabelle}
	\label{tab:uebung2}
	\renewcommand{\arraystretch}{1}
	\scriptsize
	%	\rotatebox{90}{
		\begin{tabularx}{\textwidth}{|Z|Z|c|c|c|}
			\hline
			&&\textbf{Ausgangsszenario}&\multicolumn{2}{c|}{\textbf{Szenario 1.1}}\\\hline
			&Zinssatz & Invest.\,100\,\% &Invest.\,100\,\%&  Invest.\,60\,\%\\
			\hline
			\hline
			\multirow{3}{*}{\shortstack[c]{ \textbf{LCoH }\\ \textbf{15 Jahre}\\in \euro/MWh}} & 3\,\% & 80	&90&100
			\\
			\cline{2-5}
			& 5\,\% & 90		&	100	&	110
			\\
			\cline{2-5}
			& 8\,\% & 100		&	110	&	120
			\\
			\hline		\end{tabularx}
		%}
\end{table}


\begin{equation}
	\label{eq:sum}
	min \sum_{t}^{8760} (K^{var. Gaskessel}_{t} + K^{var.W{"a}rmepumpe}_{t})
\end{equation}


\begin{align}
	&\dot{Q}^{Speicher,min} \leq \dot{Q}^{laden}_{t} \leq \dot{Q}^{Speicher,max}&\forall t \in T \label{eq:opt_grenze_beladen_speicher}\\
	&\dot{Q}^{Speicher,min} \leq \dot{Q}^{entladen}_{t} \leq \dot{Q}^{Speicher,max} &\forall t \in T \label{eq:opt_grenze_entladen_speicher}\\
	&Q^{Speicher}_{t=0} = Q^{init, sp} \cdot Q^{Kapzit \ddot{a} t}& \label{eq:opt_start_speicher}\\
	&Q^{Speicher}_{t=8760} \geq Q^{init, sp} \cdot Q^{Kapzit \ddot{a} t} & \label{eq:opt_end_speicher}
\end{align}
