%\documentclass[12pt,oneside,listof=totoc,paper=a4,headings=small]{scrbook}
\documentclass[12pt, oneside, listof=nochaptergap, listof=totoc, listof=entryprefix, paper=a4, headings=small, bibliography=totoc, numbers=noenddot]{scrbook}
%numbers = no point behinder chapter etc
% -----------------------------------
\AfterTOCHead[lof]{\def\autodot{:}}% macht : in Abbildungsverzeichnis
\AfterTOCHead[lot]{\def\autodot{:}}
% Allgemeine Formatierungen
\usepackage[ngerman]{babel}
%\usepackage[english]{babel}
\usepackage[utf8]{inputenc} 	
\usepackage[T1]{fontenc}
\usepackage[babel=true, german=quotes]{csquotes}
\usepackage{url}
%\usepackage{listings}
\usepackage[hidelinks]{hyperref} 			
\usepackage{enumerate}
\usepackage{setspace}
%\usepackage{parskip}


\usepackage[intoc, german]{nomencl}
\usepackage{ifthen}
\renewcommand{\nomname}{Nomenklatur}
\renewcommand{\nomgroup}[1]{%
\ifthenelse{\equal{#1}{A}}{\item[\textbf{Abkürzungen}]}{%
\ifthenelse{\equal{#1}{S}}{\item[\textbf{Variablen \& Symbole}]}{%
\ifthenelse{\equal{#1}{V}}{\item[\textbf{Variablen}]}{%
%\ifthenelse{\equal{#1}{G}}{\item[\textbf{Greek symbols}]}{}}}
}}}}
\setlength{\nomlabelwidth}{1.5cm}
\makenomenclature
%\usepackage{colortbl}
%\usepackage{array}
%\usepackage{color}
%\usepackage{adjustbox}
%\usepackage{multirow}
\usepackage{wrapfig}
\usepackage[toc, page]{appendix}
\usepackage{enumitem} %abstand bei itemize einstellbar [itemsep=0pt]

\setlength\parindent{0pt} %verhindert das Einrücken am Anfang

%%%%% Kopfzeile ändern 
% Quelle vor dem Punkt
%%%%%

% Kopfzeile
\usepackage[headsepline,manualmark]{scrlayer-scrpage} %plainheadsepline
\clearpairofpagestyles
\ohead{\pagemark} % * sorgt für Angaben auch auf Seiten mit Überschrift
\ihead{\headmark}
\automark{chapter}
\pagestyle{scrheadings}
% Seitenspiegel
\usepackage[left=35mm,right=15mm,top=30mm,bottom=25mm]{geometry}

%% Literatur
%\usepackage[style = numeric, labelnumber, defernumbers = true, backend = biber, autocite=inline, sorting = none]{biblatex}
\usepackage[backend=biber,style=numeric, sorting=none]{biblatex}
\bibliography{./literatur/quellen_citavi}
%  rm -rf `biber --cache`   clear cache
\usepackage{amsmath}
\usepackage{amssymb}

% Grafiken
\usepackage{graphicx} 				
%\graphicspath{{./abbildungen/}}  
%\usepackage[labelfont=footnotesize, textfont=sf, textfont=footnotesize]{caption}      
%\renewcommand{\thefigure}{\arabic{figure}} % Aufzählung bei Caption nur die Nummer ohne Kapitel 
\graphicspath{{./abbildungen/}}
\usepackage[labelfont=footnotesize, textfont=sf, textfont=footnotesize]{caption}
\captionsetup[table]{belowskip=10pt} %abstand unter der Tabellenüberschrift 

%Infos zu Caption https://www.namsu.de/Extra/pakete/Caption.html 
\usepackage{subfigure}   
\usepackage{float}
%change caption

%\counterwithout{figure}{chapter}%Aufzählung über ganzes Dokument
%\counterwithout{table}{chapter} %Aufzählung über ganzes Dokument
%\counterwithout{equation}{chapter} 
%\renewcommand{\thetable}{\arabic{table}}
%\renewcommand{\thefigure}{\arabic{figure}} % Aufzählung bei Caption nur die Nummer ohne Kapitel 

\usepackage{lscape} %use for turn page xltabular
%\begin{landscape}
%\end{landscape}

% Tabellen
\usepackage{multirow}
\usepackage{makecell}
\usepackage{xltabular} %use for long tables
\usepackage{tabularx} %makes table over hole page begin{tabularx}{\textwidth}{|X|}
\usepackage{float} %fügt neue figure optionen hinzu z.B. [H]
\usepackage[table]{xcolor} %\rowcolor{lightgray} make colors in table
%\usepackage{spreadtab}

% Koma-Script Kompatibilität
%\usepackage{scrhack}

% Überschriften
%\usepackage{titlesec}
%\titleformat{\section}
%{\normalfont\Large\bfseries}{\thesection}{1em}{}
%\titleformat{\subsection}
%{\normalfont\large\bfseries}{\thesubsection}{1em}{}
%\titleformat{\subsubsection}
%{\normalfont\normalsize\bfseries}{\thesubsubsection}{1em}{}
%\titleformat{\paragraph}[runin]
%{\normalfont\normalsize\bfseries}{\theparagraph}{1em}{}
%\titleformat{\subparagraph}[runin]
%{\normalfont\normalsize\bfseries}{\thesubparagraph}{1em}{}
%\usepackage{todonotes}
\usepackage{color,soul}%nomenclature
\usepackage{eurosym}
% Das Dokument selbst mit seinen Bestandteilen
% --------------------------------------------
\begin{document}
\begin{spacing}{1.3}
\frontmatter 
% ----------------------------------------------
%    % Titelseite soll keine Kopf oder Fußzeile haben
\thispagestyle{empty}

%logo normal
\vspace*{-20mm}
\begin{center}
   \includegraphics[width=0.8\textwidth]{}
\end{center}
\vspace*{2cm}
%logo seitlich
begin{textblock*}{80pt}(18cm, 2cm)  % {width of the block}(x,y position)
	\includegraphics[width=\linewidth,,keepaspectratio]{}
\end{textblock*}

% Alle Elemente sollen zentriert sein
\begin{center}
% Art der Arbeit => (Bachelorarbeit, Masterarbeit, Seminararbeit)
{\Large \bfseries Projektarbeit im Fach\\} 
{\Large \bfseries \glqq XXXXXX\grqq \\} 

\vspace*{1cm}

{\large Studiengang XXXXX (M. Sc.)\\[1mm]}

\vspace{1cm}

% Titel der Arbeit 
{\Large \bfseries Thematik:\\}
{\Large \bfseries XXXXX \\}


\vspace{1.5cm}

% Name des/der Autors/Autoren
{\large VORNAME NACHNAME}\\[5mm]
{\large Matrikelnummer: XXXXXXX}\\[10mm]

\end{center}
\vfill

% Aufgabensteller, Kontaktdaten und Abgabetermin
\begin{center}
\parbox{120mm}{
\begin{tabbing}
Aufgabensteller/Prüfer \hspace{.7cm} \=  Prof. Dr. XXXX\\
Arbeit vorgelegt am                  \> XX.XX.XXXX \\
durchgeführt am                  \> XXXXXX\\[4mm]
% falls der praktische Teil der Arbeit in einer Firma durchgeführt wurde.
\end{tabbing}
}
\end{center}

 			
%    \newpage
\thispagestyle{empty}

% Bitte hier keine Änderungen vornehmen, sondern vollständig handschriftlich ausfüllen
{\Large \textbf{Sperrvermerk}}\\ 

\vspace*{5mm}

\noindent
Die nachfolgende Arbeit enthält vertrauliche Informationen und Daten der Firmen XXXX sowie XXXX. Veröffentlichungen und Vervielfältigungen - auch nur auszugsweise oder in elektronischer Form - sind ohne ausdrückliche schriftliche Genehmigung der genannten Firmen nicht gestattet. Die Sperrfrist gilt bis XX.XX.XXXX. Die Arbeit darf bis zum Ablauf der Sperrfrist nur für Prüfungszwecke verwendet werden.

\vspace*{20mm}

{\Large \textbf{Eigenständigkeitserklärung}}\\ 

\vspace*{5mm}

\noindent
Hiermit versichere ich, dass ich die vorliegende Arbeit selbstständig angefertigt, nicht anderweitig für Prüfungszwecke vorgelegt, alle benutzten Quellen und Hilfsmittel angegeben sowie wörtliche und sinngemäße Zitate gekennzeichnet habe.
\vspace{2cm}

\noindent
Kempten, den XX. Month XXXX
\hspace*{2cm}%
\dotfill\\
\hspace*{8.5cm}%
\textit{Unterschrift des Verfasser/in}

\vspace*{20mm}

\noindent  {\Large \textbf{Ermächtigung}}\\ 

\vspace*{5mm}

\noindent
Hiermit ermächtige ich die Hochschule Kempten zur Veröffentlichung der Kurzzusammen\-fassung (Abstract) meiner Arbeit, zum Beispiel auf gedruckten Medien oder auf einer Internetseite.
\vspace{2cm}

\noindent
Kempten, den XX. Month XXXX
\hspace*{2cm}%
\dotfill\\
\hspace*{8.5cm}%
\textit{Unterschrift der Verfasser/in}

%    \include{./bestandteile/abstract} 
   \tableofcontents 					           
    \clearpage
%    \include{./bestandteile/nomenclature}
    \printnomenclature
    \listoffigures  					 	       
    \clearpage
    \listoftables						           
    \clearpage
    % run this in terminal "makeindex main.nlo -s nomencl.ist -o main.nls"
\mainmatter 	
	\renewcommand{\arraystretch}{1.3} %make table bigger
    %TODO Beispiele zu Ende sowie Vorlage BA überarbeiten, Main Datei überarbeiten, Bilder mit ChatGPT erstellen

\chapter{Einführung in LaTeX}
	\label{chap:einfuehrung}
	
	
\section{Schreiben des ersten Textes}
	\label{sec:erster_text}
	
	
\section{Einbinden von Objekten/Paketen}
	\label{sec:objekte}

	
\subsection{Die erste Grafik einfügen}
	\label{sec:grafik}
\begin{figure}[H]
	\centering
	%	\includegraphics[width=0.9\linewidth]{einbindung_beispielgrafik.pdf}
	\caption[Titel fürs Inhaltsverzeichnis]{Titel unter Grafik mit Quelle}
	\cite{agfw_leitfaden_wp}
	\label{fig:grafik_jpg}
\end{figure}

\begin{figure}[H]
	\centering
	%	\includegraphics[width=0.9\linewidth]{einbindung_beispielgrafik.pdf}
	\caption[Titel fürs Inhaltsverzeichnis]{Titel unter Grafik mit Quelle}
	\cite{agfw_leitfaden_wp}
	\label{fig:grafik_pdf}
\end{figure}
	
\subsection{Aufzählungen füge ich hier ein}
	\label{sec:aufzaehlung}

\begin{itemize}[noitemsep, label=+]
	\item Wärmepumpe
	\item Gaskessel 
	\item BHKW
	\item Gasturbine
\end{itemize}

\begin{enumerate}[noitemsep, label=\alph*]
	\item Wärmepumpe
	\item Gaskessel 
	\item BHKW
	\item Gasturbine
\end{enumerate}
	
\subsection{Tabellen können kompliziert sein}
	\label{sec:tabellen}
\begin{table}[H]
	\centering
	\caption{Dies ist die zweite Tabelle}
	\label{tab:uebung1}
	\begin{tabular}{l|c|r}
		Bezeichnung&Werte&Einheit \\\hline
		Gaskraftwerk&10&MW\\
		Wärmepumpe&5&MW\\
	\end{tabular}
\end{table}	

\begin{table}[H]
	\newcolumntype{Z}{>{\centering \arraybackslash}X}
	\caption{Dies ist die zweite Tabelle}
	\label{tab:uebung2}
	\renewcommand{\arraystretch}{1}
	\scriptsize
	%	\rotatebox{90}{
		\begin{tabularx}{\textwidth}{|Z|Z|c|c|c|}
			\hline
			&&\textbf{Ausgangsszenario}&\multicolumn{2}{c|}{\textbf{Szenario 1.1}}\\\hline
			&Zinssatz & Invest.\,100\,\% &Invest.\,100\,\%&  Invest.\,60\,\%\\
			\hline
			\hline
			\multirow{3}{*}{\shortstack[c]{ \textbf{LCoH }\\ \textbf{15 Jahre}\\in \euro/MWh}} & 3\,\% & 80	&90&100
			\\
			\cline{2-5}
			& 5\,\% & 90		&	100	&	110
			\\
			\cline{2-5}
			& 8\,\% & 100		&	110	&	120
			\\
			\hline		\end{tabularx}
		%}
\end{table}

	
\subsection{Formeln kann ich auch gebrauchen}
	\label{sec:formeln}
\begin{equation}
	\label{eq:sum}
	min \sum_{t}^{8760} (K^{var. Gaskessel}_{t} + K^{var.W{"a}rmepumpe}_{t})
\end{equation}
	
\begin{align}
	&\dot{Q}^{Speicher,min} \leq \dot{Q}^{laden}_{t} \leq \dot{Q}^{Speicher,max}&\forall t \in T \label{eq:opt_grenze_beladen_speicher}\\
	&\dot{Q}^{Speicher,min} \leq \dot{Q}^{entladen}_{t} \leq \dot{Q}^{Speicher,max} &\forall t \in T \label{eq:opt_grenze_entladen_speicher}\\
	&Q^{Speicher}_{t=0} = Q^{init, sp} \cdot Q^{Kapzit \ddot{a} t}& \label{eq:opt_start_speicher}\\
	&Q^{Speicher}_{t=8760} \geq Q^{init, sp} \cdot Q^{Kapzit \ddot{a} t} & \label{eq:opt_end_speicher}
\end{align}


\section{Hier lerne ich zitieren}
	\label{sec:zitieren}
	
	
\chapter{Tipps und Tricks}
	\label{chap:tipps}
	
Hier binde ich einen Link ein ...\href{https://forschung.hs-kempten.de/de/forschungsprojekt/482-heatshift}{\enquote{ HeatSHIFT}}
CO\textsubscript{2}-Kosten.

% ----------------------------------------------
%    \phantomsection
%    \addcontentsline{toc}{chapter}{Literatur}
    \sloppy
    \printbibliography
 \backmatter 
	\appendix
%	\chapter{Anhang}
\label{appendix:A}
\pagenumbering{roman}
\setcounter{page}{9}
\renewcommand\thesection{\Alph{section}}
%\renewcommand\thesubsection{\Alph{subsection}}
\renewcommand{\thefigure}{A.\arabic{figure}}
\renewcommand{\thetable}{A.\arabic{table}}

%remove section from toc with \tocless
\newcommand{\nocontentsline}[3]{}
\newcommand{\tocless}[2]{\bgroup\let\addcontentsline=\nocontentsline#1{#2}\egroup}


\tocless\section{Inhalt digitaler Anhang}


\tocless\section{Tabellen}



\end{spacing}
\end{document}

%\documentclass[12pt,oneside,listof=totoc,paper=a4,headings=small]{scrbook}
%\usepackage{package}
%...
%
%\begin{document}
%	\frontmatter 
%	% ----------------------------------------------
%	\include{Titel} 			
%	\tableofcontents 					           
%	\clearpage
%	\include{nomenclature}
%	\printnomenclature
%	\listoffigures  					 	       
%	\clearpage
%	\listoftables						           
%	\clearpage
%	% ----------------------------------------------
%	\mainmatter 	
%	\renewcommand{\arraystretch}{1.3} 
%	\include{kapitel1}
%	...
%	% ----------------------------------------------
%	\printbibliography
%	\backmatter 
%	\appendix
%	\include{anhang}
%\end{document}



